\documentclass[12pt,a4paper]{article}
\usepackage[utf8]{inputenc}
\usepackage[T1]{fontenc}
\usepackage[dutch]{babel}
\usepackage{amsmath}
\usepackage{amsfonts}
\usepackage{amssymb}
\usepackage{graphicx}
\usepackage{wrapfig}
\usepackage{enumitem}
\usepackage{hyperref}
\usepackage{gensymb}
\usepackage{siunitx}
\author{Estelle Severs, Matthias Kovacic}
\title{Afleidingen Natuurkunde}
\date{:)}
\begin{document}
    \maketitle
    \tableofcontents
    \newpage


    \section{Algemene afspraken rond dit document}


    \section{Algemene te kennen theorie}

    \subsection{Prefixen}
    \begin{center}
        \begin{tabular}{ | c | c | c | }
            \hline
            Prefix & Afkorting & Value      \\
            \hline
            Giga   & G         & $10^{9}$   \\
            Mega   & M         & $10^{6}$   \\
            Kilo   & k         & $10^{3}$   \\
            Hecto  & h         & $10^{2}$   \\
            Deka   & da        & $10^{1}$   \\
            Deci   & d         & $10^{-1}$  \\
            Centi  & c         & $10^{-2}$  \\
            Milli  & m         & $10^{-3}$  \\
            Micro  & $\mu$     & $10^{-6}$  \\
            Nano   & n         & $10^{-9}$  \\
            Pico   & p         & $10^{-12}$ \\
            \hline
        \end{tabular}
    \end{center}

    \subsection{Vectoren}
    Vergeet niet je vector altijd in componenten te splitsen!

    \begin{itemize}
        \item \(A_x = A\cos\theta\) en \(A_y = A\sin\theta\)
        \item \(A = \sqrt{A_x^2 + A_y^2}\)
        \item \(\theta = \tan^{-1}(\frac{A_y}{A_x})\)
    \end{itemize}

    \subsubsection{Scalair product}
    De grote van deze vector vermenigvuldigd met de projectie van de andere vector op deze vector. Hieruit krijg je dus een scalar!!
    \[\textbf{A} \cdot \textbf{B} = AB \cos\theta\]

    \subsubsection{Vectorproduct}
    Dit product geeft altijd een vector loodrecht op beide vectoren. Deze uitkomst is te vinden met de rechterhandregel. Als je deze nog niet kent: zoekt es op op youtube ;)
    De grootte is te vinden met volgende formule:
    \[\textbf{A} \times \textbf{B} = AB\sin\theta\]

    \subsection{Pollevs}

    \begin{itemize}
        \renewcommand\labelitemi{--}
        \item Welke uitdrukking geeft het volume van een afgeknotte kegel?
        \begin{enumerate}
            [label=\alph*)]
            \item \(\pi(r_1 + r_2)\sqrt{h^2 + (r_1 - r_2)^2}\)
            \item \(2\pi(r_1 + r_2)\)
            \item \(\pi h(r_1^2 + r_1r_2 + r_2^2)\)
        \end{enumerate}
        \textit{Oplossing:} c, dit is de enige formule die een term gaat hebben tot de 3e macht en een volume is altijd van een macht 3.

        \item Voor welke van de volgende vectoren is de grootte van de vector gelijk aan een van de componenten van de vector?
        \begin{enumerate}
            [label = \alph*)]
            \item \(\vec{A} = 2\hat{\imath} + 5\hat{\jmath}\)
            \item \(\vec{B} = -3\hat{\jmath}\)
            \item \(\vec{C} = +5\hat{k}\)
            \item \(\vec{B} \text{ en } \vec{C}\)
        \end{enumerate}
        \textit{Oplossing:} c is het juiste antwoord. a kan niet omdat de grootte van de vector moet gelijk zijn aan de grote van de component. b kan niet omdat de grootte van een component niet negatief kan zijn. (na te kijken, not sure). Hieruit volgt dat d natuurlijk niet waar kan zijn.

        \item Welk van de volgende stellingen is juist, over het verband tussen \(\vec{A} \cdot \vec{B}\) en \((-\vec{A}) \cdot (-\vec{B})\)
        \begin{enumerate}
            [label=\alph*)]
            \item \(\vec{A} \cdot \vec{B} = -((-\vec{A})\cdot(-\vec{B}))\)
            \item als \(\vec{A} \cdot \vec{B} = AB\cos\theta \), dan is \((-\vec{A}) \cdot (-\vec{B}) = AB\cos(\theta + 180\degree)\)
            \item Zowel a als b is correct.
            \item Zowel a als b is fout.
        \end{enumerate}
        \textit{Oplossing:} d is het juiste antwoord.
        Of de vectoren nu in de positieve of negatieve richting staan, de hoek zal niet veranderen.
        De lengte van de vectoren zal ook gelijk blijven.

        \item Gegeven: twee vectoren $\vec{a}$ en $\vec{b}$, gelegen in het xy-vlak.
        Bepaal \(c = \vec{a} \times \vec{b}\)
        \begin{enumerate}
            [label=\alph*)]
            \item \(\vec{c} = - ab \sin(\pi/2 - \phi)\hat{k}\)
            \item \(\vec{c} = ab \cos(\phi)\hat{k}\)
            \item \(\vec{c} = ab \cos(\pi/2 - \phi)\)
            \item Geen van deze antwoorden is correct.
        \end{enumerate}
        \textit{Oplossing:} a is juist. De richting van de vector is dan -$\hat{k}$, het vectorproduct gebruikt een sinus om de grootte te bepalen en de hoek tussen $\vec{a}$ en $\vec{b}$ is 90$\degree$ - $\theta$
    \end{itemize}


    \section{Deel 1 - Mechanica}


    \section{Kinematica in 1 dimensie}
    2.1-2.6, 2.8-2.9

    \subsection{2.5: Formules bij constante versnelling}
    We nemen aan dat het initiële tijdstip in elk van deze formules altijd 0 is. \((t_{0} = 0)\).
    Vergelijking voor snelheid afleiden:
    \[\mathbf{a = \frac{dv}{dt} = constante}\]
    $\iff$ \[a dt = dv\]
    $\iff$ \[\int_{v_0}^{v} a \, dt = \int_{0}^{t} \,dv\]
    $\iff$\[v - v_0 = at\]
    $\iff$\[\mathbf{v = v_0 + at}\]
    Vergelijking voor verplaatsing afleiden:
    \[v = \frac{dx}{dt}\]
    $\iff$\[dx = v dt\]
    $\iff$\[x - x_0 = \int_{0}^{t} v \, dt\]
    $\iff$\[x - x_0 = \int_{0}^{t} (v_0 + at) \, dt\]
    $\iff$\[x - x_0 = \int_{0}^{t} v_0 \, dt + \int_{0}^{t} at \, dt\]
    $\iff$\[\mathbf{x - x_0 = v_0t + a\frac{t^2}{2}}\]
    Alternatieve vergelijking voor snelheid:
    \[\bar{v} = \frac{v_0 + v}{2} \text{en} t = \frac{v - v_0}{a}\]
    dan geldt voor de vergelijking van verplaatsing:
    \[x = x_0 + (\frac{v + v_0}{2})(\frac{v - v_0}{a})\]
    $\iff$\[x = x_0 + \frac{v^2 + v_0^2}{2a}\]
    $\iff$\[\mathbf{v^2 = v_0^2 + 2a(x - x_0)}\]


    \section{Kinematica in twee of drie dimensies}
    3.7

    \subsection{Projectiel beweging: formules}
    \begin{figure}[h]
        \centering
        \includegraphics[width=0.7\linewidth]{projectiel}
        \caption{Er zal enkel een in de verticale component een versnelling aanwezig zijn. Hierdoor verandert de snelheid enkel in de verticale component.}
        \label{projectiel}
    \end{figure}
    \begin{table}[h]
        \centering
        \begin{tabular}{|c|c|}
            \hline
            \textbf{Horizontaal}     & \textbf{Verticaal}                        \\
            \hline
            \(a_x = 0\)              & \(a_y = -g\)                              \\
            \hline
            \(v_x(t) = v_{x0}\)      & \(v_y(t) = v_{y0} - gt\)                  \\
            \hline
            \(x(t) = x_0 + v_{x0}t\) & \(y(t) = y_0 + v_{y0}t - \frac{gt^2}{2}\) \\
            \hline
        \end{tabular}
    \end{table}


    \section{Dynamica: Newton's bewegingswetten}
    4.1-4.7

    \subsection{Eerste wet: inertie}
    Een lichaam in rust (of in eenparige rechtlijnige beweging) zal in rust (eenparige rechtlijnige beweging) blijven tenzij er een uitwendige resulterende kracht inwerkt.
    \[\sum_{i}\vec{F_i} = 0 \Rightarrow \vec{a} = 0\]

    \subsection{Tweede wet: versnelling}
    Een grotere kracht op een lichaam met massa m veroorzaakt een grotere versnelling: a \textasciitilde F

    Bij een dubbele massa 2m zal eenzelfde kracht slechts een versnelling a/2 veroorzaken: a \textasciitilde 1/m

    \[\sum_{i} \vec{F_i} = \vec{F} = m\vec{a}\]

    \subsection{Derde wet: actie-reactie}
    Bij wisselwerking tussen twee lichamen is de kracht \(\vec{F_{21}}\) van lichaam 1 op lichaam 2 even groot en tegengesteld aan de kracht \(\vec{F_{12}}\) van lichaam 2 op lichaam 1.
    \[\vec{F_{12}} = -\vec{F_{21}}\]
    Deze krachten komen steeds in paren voor en werken op verschillende voorwerpen.

    \subsection{Gewicht - Gravitatie - Normaalkracht}
    Alle voorwerpen nabij het aardoppervak vallen met dezelfde versnelling $\vec{g}$.
    \[\text{Gravitatiekracht: } \vec{F_G} = m\vec{g}\]


    \section{De wetten van Newton: wrijving, cirkelbeweging, weerstandskrachten}

    \subsection{Delen in de Giancoli}
    5.1-5.3, 5.5-5.6


    \section{De zwaartekracht en de synthese van Newton}

    \subsection{Delen in Giancoli}
    6.1-6.4, 6.6


    \section{Arbeid en energie}

    \subsection{Delen in Giancoli}
    7.1, 7.3-7.4 (+14.1)


    \section{Behoud van energie}

    \subsection{Delen in Giancoli}
    8.1-8.3, 8.5, 8.8


    \section{Impuls}

    \subsection{Delen in Giancoli}
    9.1-9.2 (+36.11)


    \section{Rotatie}
    10.1, 10.4, 10.8


    \section{Impulsmoment}

    \subsection{Delen in Giancoli}
    11.3-11.4, 11.6
    
    
    \section{Pollevs}
    - Een kanonbal volgt pad B op aarde. Welk pad zou de kanonbal op de maan volgen (\(g_{maan} = 1.6m/s^2\)) indien hij op dezelfde manier uit het kanon werd afgevuurd? (zie afbeelding cursus)
    \begin{enumerate}
    	[label=\alph*)]
    	\item A
    	\item B
    	\item C
    	\item D
    \end{enumerate}
    \textit{Oplossing:} D, g is kleiner dus de versnelling naar beneden is kleiner.
    \newline
    - Welk van de volgende stellingen is het meest correct?
    \begin{enumerate}[label=\alph*]
    	\item Het is mogelijk dat een voorwerp in beweging is zonder dat er krachten op het voorwerp werken.
    	\item Het is mogelijk dat er krachten op een voorwerp inwerken zonder dat er beweging is.
    	\item A en B zijn fout.
    	\item A en B zijn correct.
    \end{enumerate}
    \textit{Oplossing:} A en B zijn beide correct. Een voorwerp A) Een voorwerp op een trein, het beweegt tegenover de aarde. B) Een voorwerp dat op een bank ligt, maar zwaartkracht werkt hier wel op in.
    \newline
    - Een voorwerp ondervindt een nettokracht wordt hiervoor versneld. Welke stelling is \textit{altijd} vorrect?
    \begin{enumerate}[label=\alph*]
    	\item Het voorwerp beweegt in de richting van de nettokracht.
    	\item De versnelling is in dezelfde richting als de snelheid.
    	\item De versnelling is in dezelfde richting als de kracht.
    	\item De snelheid van het voorwerp neemt toe.
    \end{enumerate}
    \textit{Oplossing:} d is de juiste oplossing. $\vec{F}$ en $\vec{a}$ hebben gelijke richting en zin op een constante na: \(\vec{F} = m\vec{a}\). Tegenvoorbeeld voor A en B: Het voorwerp is al in beweging en er werkt een kracht loodrecht op dat voorwerp. Tegenvoorbeeld voor C: Als versnelling in tegengestelde richting staat (vertragen).
    \newline
    - Wanneer een vlieg botst met de voorruit van een snelrijdende bus, welk voorwerp ondervindt dan de grootste impactkracht?
    \begin{enumerate}[label=\alph*]
    	\item De vlieg
    	\item De bus
    	\item Beide ondervinden dezelfde kracht.
    \end{enumerate}
    \textit{Oplossing:} Ze ondervinden beide dezelfde kracht. Dit volgt onmiddelijk uit de 3e wet van Newton: Actie-Reactie. 
    \newline
    - Wanneer een vlieg botst met de voorruit van een snelrijdende bus, welk voorwerp ondervindt dan de grootste vernselling?
    \begin{enumerate}[label=\alph*]
    	\item De vlieg
    	\item De bus
    	\item Beide ondervinden dezelfde versnelling
    \end{enumerate}
    \textit{Oplossing:} De vlieg zal de grootste versnelling ondervinden. 
    \newline
    - Je plaatst jouw fysicaboek op een houten plank. Daarna til je één uiteinde van de plank op, zodat de hoek met de tagel toeneemt. Uiteindelijk gaat het boek glijden op de plank. Wanneer je de hoek tussen de plank en de tafel constant houdt op deze waarde, zal het boek
    \begin{enumerate}[label=\alph*]
    	\item met constante snelheid bewegen.
    	\item vertragen.
    	\item vernsellen.
    	\item Geen deze antwoorden is correct. 
    \end{enumerate}
	\textit{Oplossing:} Het boek zal versnellen. Het boek beweegt pas wanneer \(F_z > F_k\). Er blijft een resulterende kracht en versnelling zijn. 
	\newline
	- Een auto (met kale banden!) rijdt doorheen een cirkelvormige bocht, met de grootste snelheid waarbij de centripetale kracht die nodig is om de auto in een cirkel te laten bewegen, net gelijk is aan de maximale statische wrijvingskracht tussen de banden en de weg. Bij punt P rijdt de auto door een plas, waardoor de staticshe wrijvingscoëfficiënt afneemt. De auto glijdt, waardoor nu kinetische wrijving op de auto werkt. De \textit{richting van deze kinetische wrijvingskracht} is
	\begin{enumerate}[label=\alph*]
		\item in dezelfde richting van de oorspronkelijke statische wrijvingskracht.
		\item in de tegenovergestelde richting van de oorspronkelijke statische wrijvingskracht.
		\item loodrecht op de oorspronkelijke statische wrijvingskracht.
		\item onder 45\degree georienteerd ten opzichte van de oorspronkelijke statische wrijvingskracht.
	\end{enumerate}
	\textit{Oplossing:} Het juiste antwoord is c. De kinetische wrijving zal wijzen naar achter tegenover de auto terwijl de statische wrijvingskracht wees naar het middelpunt van de bocht. 
	\newline
	- Captain America staat op de top van een zeer hoge berg en gooit een baseball in de horizontale richting, met een zodanige snelheid dat de baseball in een baan rond de aarde gaat bewegen. Wanneer de baseball in deze baan beweegt, is zijn versnelling
	\begin{enumerate}[label=\alph*]
		\item afhankelijk van hoe snel de baseball geworpen werd.
		\item een beetje kleiner dan \(9.81 m/s^2\).
	 	\item gelijk aan \(9.81 m/s^2\).
	 	\item gelijk aan nul omdat de bal niet op de grond valt. 
	\end{enumerate}
	\textit{Oplossing:} Het juiste antwoord is b. De gravitatieversnellling is: \(g = \frac{M_a}{R_a^2}\). De straal zal in dit geval de straal zijn van de aarde + de hoogte van de berg waardoor de valversnelling lager zal liggen dan gewoonlijk. 
	\newline
	- De gravitatiekracht uitgeoefend door de zon op de aarde, houdt de aardein haar baan rond de zon (neem aan dat de baan een perfecte cirkel is). De arbeid geleverd door de de gravitatiekracht in een zeer kort tijdsinterval, waarbij de aarde een infinitesimale verplaatsing maakt, is
	\begin{enumerate}[label=\alph*]
		\item positief
		\item negatief
		\item nul
		\item onmogelijk te bepalen
	\end{enumerate}
	\textit{Oplossing:} Het antwoord zal nul zijn. De kracht en verplaatsing staan loodrecht op elkaar. 
	\newline
	- Met een speelgoedpistool kan je pijltjes wegschieten door eerst met het pijltje de veer van het pistool in te drukken over een afstand d. Voor een tweede schot druk je de veer over een afstand 2d in. Hoeveel sneller vliegt het tweede pijltje ten opzichte van het eerste pijltje?
	\begin{enumerate}[label=\alph*]
		\item Vier maal zo snel.
		\item Twee maal zo snel.
		\item Even snel.
		\item Half zo snel.
		\item Een vierde zo snel.
	\end{enumerate}
	\textit{Oplossing:} b is de juiste oplossing hier. De veer wordt verder ingedrukt en dus zal het kogetlje sneller gaan dan de vorige keer. We passen behoud van energie toe: \(W = \frac{kx^2}{2} = \frac{k(2x^2)}{2} = 2kx^2\). Hieruit volgt: \(K = \frac{mv^2}{2} = \frac{4mv^2}{2} = 2mv^2\) en dus is de snelheid verdubbeld.
	\newline
	- Wat is de betekenis van de helling van een U(x) grafiek?
	\begin{enumerate}[label=\alph*]
		\item De grootte van de kracht op het voorwerp.
		\item Het negatieve van de grootte van de kracht. 
		\item De x-component van de kracht op het voorwerp. 
		\item Het negatieve van de x-component van de kracht op het voorwerp. 
	\end{enumerate}
	\textit{Oplossing:} De hellling is de afgeleide van de potentiële energie tegenover x, dus: \(F_x = -\frac{dU}{dx}\)
	\newline
	- Een blok met massa m wordt op een horizontaal oppervlak geschoven met beginsnelheid v. Het blok glijdt tot het tot stilstand komt door de wrijving met het oppervlak. Hetzelfde blok wordt nu geschoven op het horizontaal oppervlak, met beginssnelheid 2v. Het blok komt tot rust op een afstand (ten opzichte van de afstand in het eerste geval):
	\begin{enumerate}[label=\alph*]
		\item die gelijk is.
		\item die tweemaal zo groot is.
		\item die viermaal zo groot is.
		\item het is niet mogelijk om het verband te bepalen. 
	\end{enumerate}
	\textit{Oplossing:} c is het juiste antwoord. De wrijving blijft gelijk en er is geen potentiële energie in het systeem. \(E_k^i = \frac{mv^2}{2}\) is wat geldt voor de initiële toestand terwijl in de tweede toestand geldt: \(E_k^f = \frac{m(2v)^2}{2} = 4\frac{mv^2}{2}\)
	\newline
	- Een auto van een oud model trekt op van rust tot een snelheid v in 10 seconden. Een nieuwer, kachtiger model versnelt van rust tot 2v in dezelfde tijd. Wat is de verhouding van het vermogen van de nieuwe auto ten opzichte van de oudere auto? 
	\begin{enumerate}[label=\alph*]
		\item 0.25
		\item 0.5
		\item 1
		\item 2
		\item 4
	\end{enumerate}
	\textit{Oplossing:} de oplossing is in dit geval 4. \(P = \frac{dW}{dt}\) waarbij de tijd in beide gevallen hetzelfde is, enkel de verhouden van arbeid moet dus berekend worden. \(\frac{W_1}{W_2} = \frac{m(2v)^2}{2} / \frac{mv^2}{2} = 4\).
	\newline 
	- Een auto en een grote vrachtwagen met dezelfde snelheid botsen frontaal en haken in elkaar vast. Welk voertuig ondervindt de grootste verandering van de impulsgrootte?
	\begin{enumerate}[label=\alph*]
		\item De auto.
		\item De vrachtwagen
		\item Beide ondervinden dezelfde verandering van impulsgrootte.
		\item Het is onmogelijk dit te bepalen. 
	\end{enumerate}
	\textit{Oplossing:} Het antwoord is c. De totale impuls blijft gelijk, wat erbij komt bij de ene zal er bij de andere af gaan. Ze worden 1 systeem na de botsing, de massa neemt toe maar de snelheid blijft gelijk. Impuls van de auto zal evenveel toenemen als deze bij de vrachtwagen afneemt. 
	\newline
	- Wanneer het niet lukt om een schroef, die zeer vast zit in een plank, los te maken, neem je best een schroevendraaier die
	\begin{enumerate}[label=\alph*]
		\item langer is.
		\item dikker is.
		\item noch A, noch B maakt iets uit.
	\end{enumerate}
	\textit{Oplossing:} Het is voordeliger als de schroevendraaier dikker is, aangezien dan je momentarm groter is. 
	\newline
	- Een schaatster nadert een paal en grijpt die vast. Hierdoor draait ze in een cirkel rond de paal. Hoe groot is haar impulsmoment ten opzichte van de paal op het ogenblik dat ze zich op een afstand d van de paal bevindt, en schaatst met een snelheid v langsheen een rechte lijn die paal op een afstand a passeert?
	\begin{enumerate}[label=\alph*]
		\item Nul
		\item mvd
		\item mva
		\item Onmogelijk te bepalen. 
	\end{enumerate}
	\textit{Oplossing:} Het antwoord is c. \(\vec{L} = \vec{r} \times \vec{p} = d sin\theta m v = amv\)
	\newline
	- Je vult een emmer tot de helft met water, en vriest het in. Dan roteer je de emmer op een wrijvingsloos platform. Wanneer het ijs smelt, zal de hoeksnelheid van de emmer
	\begin{enumerate}[label=\alph*]
		\item toenemen.
		\item afnemen.
		\item constant blijven.
		\item toe - of afnemen afhankelijk van de hoeveelheid ijs die gesmolten is. 
	\end{enumerate}
	\textit{Oplossing:} b is het antwoord. Het ijs smelt dus wordt de momentarm groter. Een andere uitleg is dat het traagheidsmoment zal vergroten en dus de hoeksnelheid zal afnemen. 
    \newpage


    \section{Deel 2 - Elektriciteit}


    \section{Elektrische velden}

    \subsection{Delen in Giancoli}
    21.1-21.2, 21.4-21.11, 21.13


    \section{De wet van Gauss}

    \subsection{Delen in Giancoli}
    22.1-22.3


	\section{Elektrische potentiaal}

    \subsection{Delen in Giancoli}
    23.1-23.9


    \section{Condensatoren en diëlektrica}

    \subsection{Delen in Giancoli}
    24.2-24.6


    \section{Elektrische stroom en weerstand}

    \subsection{Delen in Giancoli}
    25.1-25.6, 25.8-25.9 (+40.7-40.10)


    \section{Gelijkstroomschakelingen}

    \subsection{Delen in Giancoli}
    26.2-26.5, 26.7
    \newpage


    \section{Deel 3 - Magnetisme}
    Dit is geen deel van het vak in het eerste jaar, maar zal je misschien van pas komen in het tweede jaar ;).
\end{document}