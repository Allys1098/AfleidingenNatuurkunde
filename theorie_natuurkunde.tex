\documentclass[12pt,a4paper]{article}
\usepackage[utf8]{inputenc}
\usepackage[T1]{fontenc}
\usepackage[dutch]{babel}
\usepackage{amsmath}
\usepackage{amsfonts}
\usepackage{amssymb}
\usepackage{graphicx}
\author{Estelle Severs, Matthias Kovacic}
\title{Afleidingen Natuurkunde}
\date{:)}
\begin{document}
	\maketitle
	\tableofcontents
	\newpage
	\section{Algemene afspraken rond dit document}
	\newpage
	\section{Deel 1 - Mechanica}
	\section{Kinematica in 1 dimensie}
	\subsection{Delen in Giancoli}
	2.1-2.6, 2.8-2.9
	\section{Kinematica in twee of drie dimensies}
	\subsection{Delen in Giancoli}
	3.7
	\section{Dynamica: Newton's bewegingswetten}
	\subsection{Delen in Giancoli}
	4.1-4.7
	\section{De wetten van Newton: wrijving, cirkelbeweging, weerstandskrachten}
	\subsection{Delen in de Giancoli}
	5.1-5.3, 5.5-5.6
	\section{De zwaartekracht en de synthese van Newton}
	\subsection{Delen in Giancoli}
	6.1-6.4, 6.6
	\section{Arbeid en energie}
	\subsection{Delen in Giancoli}
	7.1, 7.3-7.4 (+14.1)
	\section{Behoud van energie}
	\subsection{Delen in Giancoli}
	8.1-8.3, 8.5, 8.8
	\section{Impuls}
	\subsection{Delen in Giancoli}
	9.1-9.2 (+36.11)
	\section{Rotatie}
	10.1, 10.4, 10.8
	\section{Impulsmoment}
	\subsection{Delen in Giancoli}
	11.3-11.4, 11.6
	\newpage
	\section{Deel 2 - Elektriciteit}
	\newpage
	\section{Deel 3 - Magnetisme}
	Dit is geen deel van het vak in het eerste jaar, maar zal je misschien van pas komen in het tweede jaar ;).
\end{document}